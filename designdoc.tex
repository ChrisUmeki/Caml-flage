\documentclass[11pt]{article}
\usepackage{fullpage}
\title{Caml-flage Design Document}
\author{Arzu Mammadova | am2692 \\ Shea Murphy | sm967 \\ Chris Umeki | ctu3 \\ Mena Wang | mw749}

\begin{document}
\maketitle

\section{Introduction}
Caml-flage is a social networking platform that, as its title suggests, allows users to interact with each other anonymously. People start out as anonymous users and are able to post text, photos and videos and interact with each other on a public dashboard. Users tag posts with different topics to categorize discussions. Each post and reply will have three camel-shaped buttons - up-camel (like), down-camel (dislike), and double-camel (direct reply) - thereby enabling users to react to one another’s posts. 

Each account has a “points” accumulator initially set to 0. Once a direct conversation starts between two users, they will both have an interaction score that increases with each reply. Once the score reaches a certain number, the two users will no longer be anonymous to each other. That is, both of the users identities will be exposed and each user will now be able to see the other’s personal profile, i.e. text, photos, videos, etc. 


\section{System Description}


\section{System Description}

\section{Data}

\subsubsection{Data Structures}

\subsubsection{Storage Formats}

\section{External Dependencies}

\subsection{Opium}
Web toolkit 

\subsection{ReasonReact}
Build frontend in OCaml

\section{Testing}


\end{document}
